\documentclass[a4paper,norsk, 10pt]{article}
\usepackage[utf8]{inputenc}
\usepackage{verbatim}
\usepackage{listings}
\usepackage{graphicx}
\usepackage[norsk]{babel}
\usepackage{a4wide}
\usepackage{color}
\usepackage{amsmath}
\usepackage{float}
\usepackage{amssymb}
\usepackage[dvips]{epsfig}
\usepackage[toc,page]{appendix}
\usepackage[T1]{fontenc}
\usepackage{cite} % [2,3,4] --> [2--4]
\usepackage{shadow}
\usepackage{hyperref}
\usepackage{titling}
\usepackage{marvosym }
\usepackage{subcaption}
\usepackage[noabbrev]{cleveref}
\usepackage{cite}


\setlength{\droptitle}{-10em}   % This is your set screw

\setcounter{tocdepth}{2}

\lstset{language=c++}
\lstset{alsolanguage=[90]Fortran}
\lstset{alsolanguage=Python}
\lstset{basicstyle=\small}
\lstset{backgroundcolor=\color{white}}
\lstset{frame=single}
\lstset{stringstyle=\ttfamily}
\lstset{keywordstyle=\color{red}\bfseries}
\lstset{commentstyle=\itshape\color{blue}}
\lstset{showspaces=false}
\lstset{showstringspaces=false}
\lstset{showtabs=false}
\lstset{breaklines}
\title{AST4320 Oblig 2}
\author{Daniel Heinesen, daniehei}
\begin{document}
\maketitle
\section{Exercise 1}
We have the 1D window function
\begin{equation}
W(x) =
\left\{
	\begin{array}{ll}
		1  & \mbox{if } |x| < R \\
		0 & \mbox{else } 
	\end{array}
\right.
\end{equation}
We now want to find the Fourier transform of this function. We do this straight forward from the definition of the Fourier transform
\begin{equation}
\tilde{W}(k) = \int_{-\infty}^{\infty} W(x) e^{-ikx}dx.
\end{equation}
Inserting our definition of $W(x)$ we find
\begin{equation}
\tilde{W}(k) = \int_{-\infty}^{-R}0\cdot e^{-ikx}dx + \int_{-R}^{R}1\cdot e^{-ikx}dx + \int_{R}^{\infty}0\cdot e^{-ikx}dx = \int_{-R}^{R}e^{-ikx}dx
\end{equation}
\begin{equation}
= \frac{i}{k}e^{-ikx}\bigg|_{-R}^{R} = \frac{i}{k}\left(e^{-ikR} - e^{ikR}\right) = \frac{2\sin Rk}{k}.
\end{equation}
Before we plot this function we need to notice that this is a function that we need to be careful with for $k\rightarrow 0$. With the use of L'Hôpitals rule
\begin{equation}\label{eq:at_0}
 \lim_{ k\rightarrow 0} \tilde{W}(k) = \lim_{k\rightarrow 0} \frac{2 \sin Rk}{k}
= \lim_{k\rightarrow 0} \frac{2R\cos Rk}{1} = 2R. 
 \end{equation}
We now see that $\tilde{W}(k)$ is well defined across the whole real line.

\begin{figure}[H]
\caption{The plot of the Fourier Transform of the top-hat smoothing function $W(x)$. This is plotted with $R=1$.}\label{fig:fourier}
\end{figure}

We now want to find the \textit{full width at half maximum}, FWHM, for this function. We first need to find the half maximum. From fig.\ref{fig:fourier} we see that maximum is at $0$. We know from \eqref{eq:at_0} that here $\tilde{W}(k=0) = 2R$, so half maximum is $R$. For the width we just need to find for which $k$ we have $\tilde{W} = R$ and multiply it by $2$ (this is because $\tilde{W}$ is symmetric around $0$). So we first need to solve
\begin{equation}
R = \frac{2}{k_{\text{half max}}}\sin Rk_{\text{half max}}.
\end{equation}
This is difficult to solve analytical, but easy numerically. From our program we easily find that 
\begin{equation}
\text{FWHM} = 2\cdot k_{\text{half max}} = 3.77.
\end{equation}

\end{document}

