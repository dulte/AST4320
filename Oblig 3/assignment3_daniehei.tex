\documentclass[a4paper,norsk, 10pt]{article}
\usepackage[utf8]{inputenc}
\usepackage{verbatim}
\usepackage{listings}
\usepackage{graphicx}
\usepackage[norsk]{babel}
\usepackage{a4wide}
\usepackage{color}
\usepackage{amsmath}
\usepackage{float}
\usepackage{amssymb}
\usepackage[dvips]{epsfig}
\usepackage[toc,page]{appendix}
\usepackage[T1]{fontenc}
\usepackage{cite} % [2,3,4] --> [2--4]
\usepackage{shadow}
\usepackage{hyperref}
\usepackage{titling}
\usepackage{marvosym }
\usepackage{subcaption}
\usepackage[noabbrev]{cleveref}
\usepackage{cite}
\usepackage{todonotes}


\setlength{\droptitle}{-10em}   % This is your set screw

\setcounter{tocdepth}{2}

\lstset{language=c++}
\lstset{alsolanguage=[90]Fortran}
\lstset{alsolanguage=Python}
\lstset{basicstyle=\small}
\lstset{backgroundcolor=\color{white}}
\lstset{frame=single}
\lstset{stringstyle=\ttfamily}
\lstset{keywordstyle=\color{red}\bfseries}
\lstset{commentstyle=\itshape\color{blue}}
\lstset{showspaces=false}
\lstset{showstringspaces=false}
\lstset{showtabs=false}
\lstset{breaklines}
\title{AST4320 Oblig2}
\author{Daniel Heinesen, daniehei}
\begin{document}
\maketitle

\section{Exercise 2}

We would like to find the optical depth $\tau_e$ of the IGM as a function of redshift $z$. The optical depth is given as

\begin{equation}\label{eq:tau}
\tau_e(z) = c\int_0^z \frac{n_e(z) \sigma_T dz}{H(z)(1+z)},
\end{equation}
where $\sigma_t$ is the Thompson cross section, $n_e$ the electron density and $H(z)$ the Hubble parameter. Since the Universe is taken to be completely ionized, he assume that $n_e \approx \bar{n_H} = 1.9 \cdot 10^{-7} (1+z)^3 $ cm$^{-3}$. We get the Hubble parameter from the Friedmann equation

\begin{equation}
H(z) = H_0 \sqrt{\Omega_m (1+z)^3 + \Omega_r(1+z)^4 + \Omega_{\Lambda}},
\end{equation}
where $\Omega_{\Lambda} = 0.692$, $\Omega_m = 0.308$ and $\Omega_r = 0$.

We will integrate \eqref{eq:tau} from $z=0$ to $10$. We assume that $\tau_e(0) \approx 0$.


\begin{figure}[H]
\centering
\includegraphics[scale=0.7]{tau}
\caption{The optical depth of the intergalactic medium. We see that for larger redshift the optical depth increases.}
\end{figure}


\section{Exercise 3}
\subsection{a)}
We have the differential equation for an isothermal halo

\begin{equation}\label{eq:diff}
-\dfrac{k_b T}{m_{DM} r^2}\dfrac{d}{dr}r^2 \dfrac{d}{dr} \ln \rho = 4 \pi G\rho.
\end{equation}

We have an ansatz that
\begin{equation}\label{eq:ansatz}
\rho(r) = \dfrac{A}{r^2}, \qquad A = \frac{k_b T}{2\pi Gm_{DM}}.
\end{equation}

To see that this is a solution, we put \eqref{eq:ansatz} into \eqref{eq:diff}. Looking at the RHS of \eqref{eq:diff} we get 


\begin{equation}
-\dfrac{k_b T}{m_{DM} r^2}\dfrac{d}{dr}r^2 \dfrac{d}{dr} \ln \rho = -\dfrac{k_b T}{m_{DM} r^2}\dfrac{d}{dr}r^2 \dfrac{r^2}{A}\cdot \left(-\dfrac{2A}{r^3}\right)
\end{equation}
\begin{equation}
2\dfrac{k_b T}{m_{DM} r^2}\dfrac{d}{dr}r = 2\dfrac{k_b T}{m_{DM} r^2}.
\end{equation}

Thus we get

\begin{equation}
2\dfrac{k_b T}{m_{DM} r^2} = 4\pi G \dfrac{A}{r^2} \Rightarrow A = \dfrac{k_b T}{2\pi G m_{DM} }.
\end{equation}

Thus \eqref{eq:ansatz} solves \eqref{eq:diff}.


\subsection{b)}

We have that a gas in hydrostatic equilibrium 

\begin{equation}\label{eq:dp}
\dfrac{dp}{dr} = - \dfrac{GM(<r)\rho}{r^2},
\end{equation}
where $M(<r)$ is the mass within a radius $r$ and $p$ is the pressure. We can see that our isothermal gas, with the density defined in \eqref{eq:ansatz}, behaves in the similar way. We start by finding the mass, which for a spherical symmetric mass is given as
\begin{equation}\label{eq:mass}
M(<r) = 4\pi \int_0^r \rho(r') r'^2 dr' = 4\pi \int_0^r \dfrac{A}{r'^2} r'^2 dr' = 4\pi \int_0^r A  dr' = 4\pi A r.
\end{equation}
In out expression of $A$ we have the dark matter mass $m_{DM}$. Since we now have a gas, we let $m_{MD} \rightarrow m_p$, which is the proton mass. Thus we have
\begin{equation}
\rho = \dfrac{A}{r^2} = \dfrac{k_b T}{2\pi G m_p r^2}.
\end{equation}
We when use that for an isothermal gas the pressure is given as

\begin{equation}
p = \dfrac{k_b T}{m_p}\rho = \dfrac{k_b T A}{m_p r^2}.
\end{equation}

We can now take the differentiation of this with respect to $r$ and use \eqref{eq:mass} and \eqref{eq:ansatz} to find
\begin{equation}
\frac{dp}{dr} = -\dfrac{2 k_b T A}{m_p r^3} =  -\dfrac{2 k_b T}{m_p r^3} \cdot \dfrac{M(<r)}{4\pi r} = - \frac{GM(<r)}{r^2}\dfrac{k_b T}{2\pi G m_p r^2} =  - \dfrac{GM(<r)\rho}{r^2}.
\end{equation}

This is the same as for the gas in hydrostatic equilibrium from \eqref{eq:dp}.



\end{document}

